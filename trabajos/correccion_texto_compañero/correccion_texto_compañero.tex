% article example for classicthesis.sty
\documentclass[10pt,a4paper]{article} % KOMA-Script article scrartcl
\usepackage{import}
\usepackage{xifthen}
\usepackage{pdfpages}
\usepackage{transparent}
\newcommand{\incfig}[1]{%
    \def\svgwidth{\columnwidth}
    \import{./figures/}{#1.pdf_tex}
}
\usepackage{lipsum}     %lorem ipsum text
\usepackage{titlesec}   %Section settings
\usepackage{titling}    %Title settings
\usepackage[margin=10em]{geometry}  %Adjusting margins
\usepackage{setspace}
\usepackage{listings}
\usepackage{amsmath}    %Display equations options
\usepackage{amssymb}    %More symbols
\usepackage{xcolor}     %Color settings
\usepackage{pagecolor}
\usepackage{mdframed}
\usepackage[spanish]{babel}
\usepackage[utf8]{inputenc}
\usepackage{longtable}
\usepackage{multicol}
\usepackage{graphicx}
\graphicspath{ {./Images/} }
\setlength{\columnsep}{1cm}

% ====| color de la pagina y del fondo |==== %
\pagecolor{white}
\color{black}



\begin{document}
    %========================{TITLE}====================%
    \title{\rmfamily\normalfont\spacedallcaps{Correccion texto de compañeros}}
    \author{\spacedlowsmallcaps{Rodrigo Castillo}}
    \date{\today}

    \maketitle
    %=======================NOTES GOES HERE===================%
    \\
        \\ \color{blue} $ azul:  $ lo que me gustó \color{black}
        \\ \color{forestgreen} $ rojo:  $ lo que no me gustó \color{black}
        \\ \color{black} $ negro:  $ comentarios mios \color{black}
    \section{texto original de mi compañero}

        \color{red} La fotosíntesis: 6CO2 + 6H2O C6H12O6 + 6O2, es un proceso químico en el cual
        las plantas capturan la energía solar y logran convertir agua, dióxido de
        carbono y ciertos minerales en oxígeno y componentes orgánicos ricos en
        energía. \color{black} . \color{black} (esta información me parece
        innecesaria para un resumen) \color{white} .
        \\
        \color{blue} Concorde a esta definición, el elemento de la tierra no es fundamental
        para este proceso, de modo que el cultivo hidropónico, o también conocido como
        el cultivo urbano, es un método de crecer plantas cuyo objetivo es hacer uso de
        esta información y crecerlas en la ausencia de tierra, es decir, en agua o una
        solución de nutrientes. Esto puede ser de gran importancia para países con
        tierra que no es buena para cultivar, o en una escala más pequeña, a conjuntos
        urbanos con poco espacio para cultivar.  \color{black}(esta información
        me pareció útil y la idea me pareció buena, me parece que es una buena
        característica para enfocarlo en publicidad) \\

        \color{blue} El propósito de este reporte técnico es presentar el diseño y funcionalidad de
        un sistema de agricultura automático basado en el método hidropónico para así,
        solucionar el problema de la falta de espacio y tiempo para realizar una huerta
        casera. Este proyecto cuenta con un prototipo a escala de una torre hidropónica
        de cultivo, en la cual se van a medir cuatro variables en específico. Además de
        estos componentes físicos para el montaje del diseño, también cuenta con una
        interfaz de usuario digital entrelazada con una implementación de un Arduino,
        para el uso eficaz del prototipo desde una plataforma web.
        \color{black}  me parece buena idea la implementación de una aplicación
        web\\

        \color{blue} El método depende de la adquisición de datos concorde a las cuatro variables
        que se está midiendo, estas siendo: humedad, luz, temperatura y pH, \color{black}
        \color{red} adicionalmente, la información específica previa sobre el cultivo es de suma
        importancia. \color{black}  (me parece que debería esclarecer que tipo
        de plantas van a sembrar en el cultivo o no mencionar el
        tema del tipo de cultivo que se va a sembrar hasta que la desición esté
        tomada)\color{blue} Entonces, concorde a la información de los
        sensores, los valores
        obtenidos se comparan con valores predispuestos de dichas huertas, para así
        lograr que el sistema automatizado inteligente formule una reacción adecuada y
        correspondiente para el beneficio del cultivo. \color{black}  \\

        \color{blue} Este reporte técnico muestra como el uso de la tecnología con el método
        hidropónico de cultivar puede ser una solución al problema de no poder
        cultivar, ya sea por falta de espacio o tierra mala para la agricultura. No
        solamente eso, sino que también se puede llevar al cabo a mayor escala, como a
        nivel nacional o en un espacio urbano delimitado, y por ende beneficiar una
        población con la producción de comida de una forma innovadora. \color{black}  \\


        \section{Lo que me gustó:}
        Me parecieron muy positivos los beneficios que este problema podría
        tener, creo que la implementación de un cultivo hidropónico sería muy
        positivo para nuestra sociedad. Me parece que es un buen proyecto, bien
        planteado y con potencial de tener mucho impacto.


        \section{Lo que no me gustó:}
           el párrafo que habla de la fotosíntesis me pareció innecesario, creo
           que sin ese párrafo, el abstract está completo y muy adecuado

















    %=======================NOTES ENDS HERE===================%

    % bib stuff
    \nocite{*}
    \addtocontents{toc}{\protect\vspace{\beforebibskip}}
    \addcontentsline{toc}{section}{\refname}
    \bibliographystyle{plain}
    \bibliography{../Bibliography}
\end{document}
