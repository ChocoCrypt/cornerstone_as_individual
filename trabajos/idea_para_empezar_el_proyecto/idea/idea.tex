% article example for classicthesis.sty
\documentclass[10pt,a4paper]{article} % KOMA-Script article scrartcl
\usepackage{import}
\usepackage{xifthen}
\usepackage{pdfpages}
\usepackage{transparent}
\newcommand{\incfig}[1]{%
    \def\svgwidth{\columnwidth}
    \import{./figures/}{#1.pdf_tex}
}
\usepackage{lipsum}     %lorem ipsum text
\usepackage{titlesec}   %Section settings
\usepackage{titling}    %Title settings
\usepackage[margin=10em]{geometry}  %Adjusting margins
\usepackage{setspace}
\usepackage{listings}
\usepackage{amsmath}    %Display equations options
\usepackage{amssymb}    %More symbols
\usepackage{xcolor}     %Color settings
\usepackage{pagecolor}
\usepackage{mdframed}
\usepackage[spanish]{babel}
\usepackage[utf8]{inputenc}
\usepackage{longtable}
\usepackage{multicol}
\usepackage{graphicx}
\graphicspath{ {./Images/} }
\setlength{\columnsep}{1cm}

% ====| color de la pagina y del fondo |==== %
\pagecolor{black}
\color{white}



\begin{document}
    %========================{TITLE}====================%
    \title{\rmfamily\normalfont\spacedallcaps{ idea para el cornerstone }}
    \author{\spacedlowsmallcaps{Rodrigo Castillo}}
    \date{\today}

    \maketitle


     % ====| Loguito |==== %
    \includegraphics[width=0.1\linewidth]{negro_cara.png}
    %=======================NOTES GOES HERE===================%
    \section{Idea}
        \subsection{idea facil inicial}
        para el betha , lo ideal sería lograr automatizar el riego de
        diferentes plantas según sus características y su estado, para esto,
        necesitaremos diseñar distintos tipos de sensores que nos retornen
        estados de las diferentes plantas(humedad,estado de la tierra, bla bla
        bla) , segun esto, y según lo que investiguemos de los diferentes tipos
        de plantas, implementar un programa que automatice el cuidado de éstas
        \begin{figure}[ht]
            \centering
            \incfig{ima}
            \caption{ima}
            \label{fig:ima}
        \end{figure}
        \subsection{formas de escalar}
            una vez teniendo esto, podemos implementarlo remotamente podemos:
            \begin{enumerate}
                \item {manejarlo remotamente desde una interfaz web}
                \item {desde la interfaz web, hacer una aplicacion web que
                    controle diferentes sistemas de riegos para diferentes
                usuarios}
                \item {mejorar la presentación del producto}
                \item {agregar especies de plantas}

            \end{enumerate}












    %=======================NOTES ENDS HERE===================%

    % bib stuff
    \nocite{*}
    \addtocontents{toc}{\protect\vspace{\beforebibskip}}
    \addcontentsline{toc}{section}{\refname}
    \bibliographystyle{plain}
    \bibliography{../Bibliography}
\end{document}
